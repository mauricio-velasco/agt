\documentclass[11pt]{article}
\usepackage[utf8]{inputenc}
\usepackage[spanish]{babel}
\usepackage{enumitem}
\usepackage{hyperref}

\date{}

\begin{document}
\section*{Proyecto final:}
El proyecto del curso es un trabajo en grupo en el que los estudiantes explorarán algún area de la teoria de juegos algorítmica de su interes que no haya sido cubierta en clase (ó una profundización de un área que si hayamos visto). El tema mismo es una decisión del estudiante y eso es parte de la formación que aporta el proyecto.

Deben preparar, entregar y exponer un trabajo escrito siguiendo el formato utilizado en los \textit{What is?} de la AMS: \url{https://www.ams.org/notices/200511/what-is.pdf}.

Concretamente el proyecto debe seguir las siguientes reglas:

\begin{enumerate}
    \item El proyecto es un trabajo en grupo. Cada grupo debe ser de máximo 2 personas (y la nota será la misma para todos los integrantes del grupo).

    \item Es {\bf necesario} leer por lo menos tres fuentes (ya sea capítulos de libro o artículos de investigación) pertinentes a cada proyecto y el artículo debe tener bibliografía.

    \item \textbf{Entregas:} La totalidad del proyecto consiste de dos entregas (a hacer en \LaTeX, un solo documento por grupo en cada entrega):
    
    \begin{enumerate}[label=\alph*)]
        \item \textbf{Entrega 1: Semana 4} Entregar el título del proyecto en que van a trabajar, los integrantes del grupo y un párrafo con una descripción del problema en que van a trabajar y de por que les parece interesante.        
        \item \textbf{Entrega 2 Semana 7:} La entrega final consiste de dos partes:
        \begin{enumerate}
            \item Un documento en \LaTeX{} explicando los resultados del proyecto. Debe ser de a lo más 10 páginas de longitud, siguiendo el formato de los \textit{What is?} de la AMS: \url{https://www.ams.org/journals/notices/201405/rnoti-p492.pdf}
            \item Una charla de 20 minutos con el objetivo de explicar los resultados del proyecto.            
        \end{enumerate}
            Los estudiantes deben asistir a todas las charlas de sus compañeros el día de las exposiciones finales.
    \end{enumerate}
\end{enumerate}
El uso de chatbots está permitido únicamente como herramienta de apoyo. Todo contenido generado debe ser revisado, comprendido y adaptado por el estudiante. El plagio o la entrega de trabajos generados mediante inteligencia artificial será considerado falta académica grave.

\end{document}
